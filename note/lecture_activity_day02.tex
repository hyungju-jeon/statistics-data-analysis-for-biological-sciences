\documentclass[a4paper,11pt]{exam}

\usepackage[T1]{fontenc}
\usepackage[top=2cm, bottom=2cm, left=3cm, right=3cm]{geometry}
\usepackage{amsmath,amssymb}
\usepackage{amsthm}
\usepackage{hyperref}
\usepackage{bm} % bold greek
\usepackage{tcolorbox}
\usepackage{xcolor}
\usepackage{forloop}
\usepackage{amsbsy}
\usepackage{tikz}
\usetikzlibrary{trees,arrows}
%\usepackage[framed,numbered]{matlab-prettifier}
%\usepackage{filecontents}
\usepackage[numbers,sort&compress,comma]{natbib}

\usepackage{graphicx}
\DeclareGraphicsExtensions{.pdf,.png,.jpg,.mps,.eps,.ps}
\graphicspath{{figs/}}

\newtheorem{theorem}{Theorem}
\newtheorem{lemma}[theorem]{Lemma}
\newtheorem{proposition}[theorem]{Proposition}

\newcommand*{\clapp}[1]{\hbox to 0pt{\hss#1\hss}}
\newcommand*{\mat}[1]{\boldsymbol{\mathrm{#1}}}
\newcommand*{\subdims}[3]{\clapp{\raisebox{#1}[0pt][0pt]{$\scriptstyle(#2 \times #3)$}}}
\fboxrule=1pt

\newcommand{\defvec}[1]{\expandafter\newcommand\csname v#1\endcsname{{\mathbf{#1}}}}
\newcommand{\defrv}[1]{\expandafter\newcommand\csname rv#1\endcsname{{\mathbf{#1}}}}
\newcounter{ct}
\forLoop{1}{26}{ct}{
	\edef\letter{\alph{ct}}
	\expandafter\defvec\letter
}
% distinguish random vector
\forLoop{1}{26}{ct}{
	\edef\letter{\Alph{ct}}
	\expandafter\defrv\letter
}
\forLoop{1}{26}{ct}{
	\edef\letter{\alph{ct}}
	\expandafter\defrv\letter
}
\newcommand{\fundamentalSolution}{\boldsymbol{\Phi}}
\newcommand{\tvv}[1]{\mathbf{\tilde{#1}}} % tilde vectors
\newcommand{\hvv}[1]{\mathbf{\hat{#1}}} % hat vectors

\newcommand{\braket}[2]{\ensuremath{\langle#1\mid#2\rangle}}
\newcommand{\bra}[1]{\ensuremath{\langle#1|}}
\newcommand{\ket}[1]{\ensuremath{|#1\rangle}}
\newcommand{\horzbar}{\rule[.5ex]{1.5em}{0.4pt}}

\newcommand{\inv}{^{-1}}
\newcommand{\trp}{{^\top}} % transpose
\newcommand{\ctrp}{{^\dagger}} % transpose
\newcommand{\tp}[1]{\ensuremath{{#1}^\top}} % transpose
\newcommand{\dm}[1]{\ensuremath{\mathrm{d}{#1}}} % dx dy dz dmu
\newcommand{\RN}[2]{\frac{\dm{#1}}{\dm{#2}}} % (Radon-Nikodym) derivative
\newcommand{\PD}[2]{\frac{\partial #1}{\partial #2}} % partial derivative
\newcommand{\norm}[1]{\ensuremath{\Vert{#1}\Vert}}
\providecommand{\abs}[1]{\lvert#1\rvert}
\newcommand{\field}[1]{\ensuremath{\mathbb{#1}}}
\newcommand{\Kfield}{\field{K}}
\newcommand{\reals}{\field{R}}
\newcommand{\complex}{\field{C}}
\newcommand{\integers}{\field{Z}}
\newcommand{\naturalNumbers}{\field{N}}

\DeclareMathOperator*{\E}{\mathbb{E}} % expectation
\DeclareMathOperator*{\var}{var}
\DeclareMathOperator*{\cov}{cov}
\DeclareMathOperator*{\diag}{diag}
\newcommand{\system}[2]{\mathcal{#1}\left[ #2 \right]}
\DeclareMathOperator*{\tr}{tr} % trace
\newcommand{\indicator}[1]{\mathbf{1}_{#1}} % indicator function
\newcommand{\onehalf}{\frac{1}{2}}

\newcommand{\iid}{{\it i.i.d.}\xspace}
\newcommand{\iidsample}{\stackrel{iid}{\sim}}
\newcommand{\normalDist}{\mathcal{N}}
\newcounter{homework}
\newcommand{\homework}{\stepcounter{homework}\textcolor{violet}{\textbf{Homework \thehomework:}~}}
\newcommand{\funfact}{\textbf{Fun Fact:}~}
\newcommand{\myparagraph}[1]{\paragraph{#1}\mbox{}\\}


\pagestyle{headandfoot}
\runningheadrule
\runningheader{INDP 2023}{Statistics and Data Analysis for Biological Sciences}{Il Memming Park}

\title{Day 2 - The ubiquitous bell-shaped curve}
\author{Il Memming Park, Hyungju Jeon}

\begin{document}
\maketitle
\begin{tcolorbox}[colback=black!1!,title=Learning Objective 2]
	Judge a distribution by its standard deviation or variance. Know what information is lost with summary statistics. Apply central limit theorem in the context of experiments. Scientifically interpret linear regression results.
\end{tcolorbox}
\section{Discrete Probability Distribution}
\subsection{Binomial Distribution}

\myparagraph{The problem of false positives} No diagnostic test is perfect in the real world.
For example, in a flow cytometry test, we may have issues with unspecific binding and specific cross-reactivity with off targets resulting in false positives, as well as lack of reactivity with the stated target antigen resulting in false negatives. Mathematically, these can be described using \textbf{sensitivity} and \textbf{specificity}.
\begin{itemize}
	\item Sensitivity is the probability of a positive test result, conditioned on the individual truly being positive.
	\item Specificity is the probability of a negative test result, conditioned on the individual truly being negative.
\end{itemize}

Suppose we have a test designed to identify the presence of a pathogen in patient blood samples. The sensitivity of the test is 80\% and the specificity is 99\%.

\begin{questions}
	\question Suppose that 10\% of the population is infected with the pathogen.
	\begin{parts}
		\part What is the probability that a randomly selected patient will be tested positive?
		\vspace{\stretch{0.3}}
		\part If a randomly selected patient is tested and the test result is \textbf{positive}, what is the probability that the patient, in fact, has the pathogen?
		\vspace{\stretch{0.3}}
		\part If a randomly selected patient is tested and the test result is \textbf{negative}, what is the probability that the patient, in fact, has the pathogen?
		\vspace{\stretch{0.3}}
	\end{parts}
	\question Suppose now that 0.1\% of the population is infected with the pathogen. Does the probabilty above change? If so, how?
\end{questions}

\subsection{DefrivingBinomial Distribution}
\tikzstyle{level 1}=[level distance=10mm, sibling distance=60mm]
\tikzstyle{level 2}=[level distance=10mm, sibling distance=30mm]
\tikzstyle{level 3}=[level distance=10mm, sibling distance=15mm]
\tikzstyle{level 4}=[level distance=10mm]
\begin{center}
	\begin{tikzpicture}[grow=down,->,>=angle 60]
		\node {Start}
		child {node {H}
				child {node{HH}
						child {node{HHH}
								child[-] {node{3}}
							}
						child {node {HHT}
								child[-] {node{2}}
							}
					}
				child {node{HT}
						child {node{HTH}
								child[-] {node{2}}
							}
						child {node {HTT}
								child[-] {node{1}}
							}
					}
				edge from parent
				node[above] {$p$}
			}
		child {node {T}
				child {node{TH}
						child {node{THH}
								child[-] {node{2}}
							}
						child {node {THT}
								child[-] {node{1}}
							}
					}
				child {node {TT}
						child {node{TTH}
								child[-] {node{1}}
							}
						child {node {TTT}
								child[-] {node{0}}
							}
					}
				edge from parent
				node[above] {$1-p$}
			};
		\node at (-80mm,-40mm) { Number of heads (X)};
	\end{tikzpicture}
\end{center}


\homework Simulate Monty Hall puzzle and answer questions in \\ \verb|day01_activity01_montyhall.ipynb|.

\vspace{5em}

\homework Following the instructions in \verb|day01_activity02_twoChildren.ipynb|, complete a code for Two-Children puzzle simulation and answer questions
\end{document}
